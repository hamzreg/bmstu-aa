\chapter{Технологическая часть}

В данном разделе будут указаны средства реализации, будут представлены листинги кода, а также функциональные тесты.

\section{Средства реализации}

Реализация данной лабораторной работы выполнялась при помощи языка программирования Python \cite{python}. Выбор ЯП обусловлен простотой синтаксиса, большим числом библиотек и эффективностью визуализации данных.

Замеры времени проводились при помощи функции process\_time из библиотеки time \cite{python-time}.

\section{Листинги кода}

Стандартный алгоритм, алгоритм Винограда и оптимизированный алгоритм Винограда умножения матриц приведены в листингах \ref{lst:standard}-\ref{lst:optimized}.

\begin{center}
\captionsetup{justification=raggedright,singlelinecheck=off}
\begin{lstlisting}[label=lst:standard,caption=Стандартный алгоритм умножения матриц]
def standard_mult(A, B):
    n = len(A)
    m = len(B[0])
    p = len(A[0])

    C = [[0] * m for i in range(n)]

    for i in range(n):
        for j in range(m):
            for k in range(p):
                C[i][j] += A[i][k] * B[k][j]

    return C
\end{lstlisting} 
\end{center}
\clearpage
\begin{center}
\captionsetup{justification=raggedright,singlelinecheck=off}
\begin{lstlisting}[label=lst:winograd,caption=Алгоритм Винограда умножения матриц]
def winograd_mult(A, B):
    n = len(A)
    m = len(B[0])
    p = len(A[0])

    C = [[0] * m for _ in range(n)]

    row_factors = [0] * n

    for i in range(n):
        for j in range(p // 2):
            row_factors[i] = row_factors[i] + \
                             A[i][2 * j] * A[i][2 * j + 1]

    column_factors = [0] * m

    for i in range(m):
        for j in range(p // 2):
            column_factors[i] = column_factors[i] + \
                                B[2 * j][i] * B[2 * j + 1][i]

    for i in range(n):
        for j in range(m):
            C[i][j] = -row_factors[i] - column_factors[j]

            for k in range(p // 2):
                C[i][j] = C[i][j] + (A[i][2 * k + 1] + B[2 * k][j]) * \
                                    (A[i][2 * k] + B[2 * k + 1][j])
    
    if p % 2 != 0:
        for i in range(n):
            for j in range(m):
                C[i][j] = C[i][j] + A[i][p - 1] * B[p - 1][j]
    
    return C
\end{lstlisting}
\end{center}
\clearpage
\begin{center}
\captionsetup{justification=raggedright,singlelinecheck=off}
\begin{lstlisting}[label=lst:optimized,caption=Оптимизированный алгоритм Винограда умножения матриц]
def optimized_winograd_mult(A, B):
    n = len(A)
    m = len(B[0])
    p = len(A[0])

    C = [[0] * m for _ in range(n)]

    row_factors = [0] * n

    for i in range(n):
        for j in range(1, p, 2):
            row_factors[i] += A[i][j] * A[i][j - 1]

    column_factors = [0] * m

    for i in range(m):
        for j in range(1, p, 2):
            column_factors[i] += B[j][i] * B[j - 1][i]

    flag = p % 2

    for i in range(n):
        for j in range(m):
            C[i][j] = -(row_factors[i] + column_factors[j])

            for k in range(1, p, 2):
                C[i][j] += (A[i][k - 1] + B[k][j]) * \
                                    (A[i][k] + B[k - 1][j])
    
            if flag:
                C[i][j] += A[i][p - 1] * B[p - 1][j]
    
    return C
\end{lstlisting}
\end{center}

\section{Функциональные тесты}

В таблице \ref{tbl:func_test} приведены функциональные тесты для функций, реализующих алгоритмы умножения матриц. Все тесты пройдены успешно.

\begin{table}[h]
	\begin{center}
		\begin{threeparttable}
		\captionsetup{justification=raggedright,singlelinecheck=off}
		\caption{\label{tbl:func_test} Функциональные тесты}
		\begin{tabular}{|c@{\hspace{7mm}}|c@{\hspace{7mm}}|c@{\hspace{7mm}}|c@{\hspace{7mm}}|c@{\hspace{7mm}}|c@{\hspace{7mm}}|}
			\hline
			Матрица A & Матрица B & Ожидаемый результат \\ 
			\hline
			$\begin{pmatrix}
				&
			\end{pmatrix}$ &
			$\begin{pmatrix}
				&
			\end{pmatrix}$ &
			Сообщение об ошибке \\ \hline

			$\begin{pmatrix}
				&
			\end{pmatrix}$ &
			$\begin{pmatrix}
				1 & 2\\
				3 & 4\\
				5 & 6
			\end{pmatrix}$ &
			Сообщение об ошибке \\ \hline

			$\begin{pmatrix}
				1 & 0 & 1
			\end{pmatrix}$ &
			$\begin{pmatrix}
				-1 & 0 & -1
			\end{pmatrix}$ &
			Сообщение об ошибке \\ \hline

			$\begin{pmatrix}
				1 & 2 & 3 \\
				4 & 5 & 6 \\
				7 & 8 & 9
			\end{pmatrix}$ &
			$\begin{pmatrix}
				-1 & -2 & -3 \\
				-4 & -5 & -6 \\
				-7 & -8 & -9
			\end{pmatrix}$ &
			$\begin{pmatrix}
				-30 & -36 & -42 \\
				-66 & -81 & -96 \\
				-102 & -126 & -150
			\end{pmatrix}$ \\ \hline

			$\begin{pmatrix}
				1 & 2 & 3
			\end{pmatrix}$ &
			$\begin{pmatrix}
				1 \\
				2 \\
				3
			\end{pmatrix}$ &
			$\begin{pmatrix}
				14 \\
			\end{pmatrix}$ \\ \hline

		\end{tabular}
		\end{threeparttable}
	\end{center}
	
\end{table}

\section{Вывод}

Были реализованы функции алгоритмов умножения матриц. Было проведено функциональное тестирование указанных функций.
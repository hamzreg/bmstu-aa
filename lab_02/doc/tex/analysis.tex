\chapter{Аналитическая часть}

В данном разделе будут описаны алгоритмы умножения матриц: стандартный, Винограда и оптимизация алгоритма Винограда.

\section{Стандартный алгоритм матричного умножения}

Пусть даны две матрицы A и B размерности $N \cdot P$ и $P \cdot M$ соответсвенно:

\begin{equation}
	A_{NP} = \begin{pmatrix}
		a_{11} & a_{12} & \ldots & a_{1m}\\
		a_{21} & a_{22} & \ldots & a_{2m}\\
		\vdots & \vdots & \ddots & \vdots\\
		a_{l1} & a_{l2} & \ldots & a_{lm}
	\end{pmatrix},
	\quad
	B_{PM} = \begin{pmatrix}
		b_{11} & b_{12} & \ldots & b_{1n}\\
		b_{21} & b_{22} & \ldots & b_{2n}\\
		\vdots & \vdots & \ddots & \vdots\\
		b_{m1} & b_{m2} & \ldots & b_{mn}
	\end{pmatrix},
\end{equation}

Тогда матрица C размерностью N*M:

\begin{equation}
	C_{NM} = \begin{pmatrix}
		c_{11} & c_{12} & \ldots & c_{1n}\\
		c_{21} & c_{22} & \ldots & c_{2n}\\
		\vdots & \vdots & \ddots & \vdots\\
		c_{l1} & c_{l2} & \ldots & c_{ln}
	\end{pmatrix},
\end{equation}

в которой:

\begin{equation}
	\label{eq:M}
	c_{ij} =
	\sum_{r=1}^{P} a_{ir}b_{rj} \quad (i=\overline{1,N}; j=\overline{1,M})
\end{equation}

называется их произведением.

Для вычисления произведения двух матриц, каждая строка первой матрицы почленно умножается на каждый столбец второй, затем подсчитывается сумма таких произведений и записывается в соответствующий элемент результирующей матрицы \cite{alg}.

\section{Алгоритм Винограда умножения матриц}

Если рассмотреть результат умножения двух матриц, то видно, что каждый элемент в нем представляет собой скалярное произведение соответствующих строки и столбца исходных матриц. Можно заметить также, что такое умножение допускает предварительную обработку, позволяющую часть работы выполнить заранее.

Рассмотрим два вектора $V = (v_1, v_2, v_3, v_4)$ и $W = (w_1, w_2, w_3, w_4)$.

Их скалярное произведение равно: $V \cdot W = v_1w_1 + v_2w_2 + v_3w_3 + v_4w_4$, что эквивалентно (\ref{for:new}):
\begin{equation}
	\label{for:new}
	V \cdot W = (v_1 + w_2)(v_2 + w_1) + (v_3 + w_4)(v_4 + w_3) - v_1v_2 - v_3v_4 - w_1w_2 - w_3w_4.
\end{equation}

Выражение в правой части формулы \ref{for:new} допускает предварительную обработку: его части можно вычислить заранее и запомнить для каждой строки первой матрицы и для каждого
столбца второй, что позволяет выполнять для каждого элемента лишь первые два умножения и последующие пять сложений, а также дополнительно два сложения. При нечетном значении размера матрицы нужно дополнительно добавить произведения крайних элементов соответствующих строк и столбцов к результату \cite{alg}.

\section{Оптимизация алгоритма Винограда умножения матриц}

Оптимизированная версия алгоритма Винограда \cite{alg} отличается:

\begin{itemize}
	\item использованием битового сдвига, вместо деления на 2;
	\item последний цикл для нечетных элементов включен в основной цикл, используя дополнительные операции в случае нечетности;
	\item замены операций сложения и вычитания на += и -= соответственно.
\end{itemize}

\section{Вывод}

Были изучены способы умножения матриц при помощи классического алгоритма и алгоритма Винограда. Также была рассмотрена оптимизация алгоритма Винограда.

Программе, реализующей данные алгоритмы, на вход будут подаваться две матрицы. Выходными данными такой программы должна быть матрица - результат умножения введенных. Программа должна работать в рамках следующих ограничений: 

\begin{itemize}
	\item элементы матрицы - целые числа;
	\item количество столбцов одной матрицы должно совпадать с количеством строк второй матрицы;
	\item должно быть выдано сообщение об ошибке при вводе пустых матриц.
\end{itemize}

Пользователь должен иметь возможность выбора алгоритма матричного умножения и вывода результата на экран. Также должны быть реализованы сравнение алгоритмов по времени работы с выводом результатов на экран и получение графического представления результатов сравнения. Данные действия пользователь должен выполнять при помощи меню.
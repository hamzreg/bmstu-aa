\chapter*{Введение}
\addcontentsline{toc}{chapter}{Введение}

Задача удобного представления и хранения информации возникает в математике, физике, экономике, статистике и программировании. Для решения этой задачи используют матрицу.

Матрица - массив, элементы которого являются массивами \cite{virt}. Эти массивы называют строками матрицы. Элементы, стоящие на одной и той же позиции в разных строках, образуют столбец матрицы.

Над матрицами можно проводить следующие операции: сложение, вычитание, транспонирование, возведение в степень. Одной из часто применяющихся операций является умножение матриц. Например, в машинном обучении реализации распространения сигнала в слоях нейронной сети базируются на умножении матриц. Так, актуальной задачей является оптимизация алгоритмов матричного умножения.

Целью данной лабораторной работы является изучение алгоритмов умножения матриц. Для достижения поставленной цели требуется выполнить следующие задачи:

\begin{itemize}
	\item изучить алгоритмы стандартного умножения матриц и алгоритм Винограда;
	\item рассмотреть оптимизацию указанного алгоритма;
	\item привести схемы изучаемых алгоритмов;
	\item провести сравнительный анализ по трудоемкости рассмотренных алгоритмов;
	\item описать используемые типы и структуры данных;
	\item описать структуру разрабатываемого программного обеспечения;
	\item определить средства программной реализации выбранных алгоритмов;
	\item реализовать разработанные алгоритмы;
	\item провести функциональное тестирование программного обеспечения;
	\item провести сравнительный анализ по времени реализованных алгоритмов;
	\item подготовить отчет о выполненной лабораторной работе.
\end{itemize}
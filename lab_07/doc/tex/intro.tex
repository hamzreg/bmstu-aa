\chapter*{Введение}
\addcontentsline{toc}{chapter}{Введение}

Одной из ключевых операций при обработке данных является поиск. Время работы алгоритмов поиска зависит от размера набора данных. Для представления данных используются различные структуры, которые удобно использовать в конкретных задачах. Одной из таких структур данных является словарь \cite{dict}. Следовательно, становится актуальной задача быстрого поиска в словаре.

Целью данной лабораторной работы является изучение алгоритмов поиска в словаре. Для достижения поставленной цели требуется выполнить следующие задачи:

\begin{itemize}
	\item изучить структуру данных - словарь;
	\item рассмотреть алгоритмы поиска в словаре: полный перебор, бинарный поиск и поиск сегментами;
	\item привести схемы изучаемых алгоритмов;
	\item описать используемые типы и структуры данных;
	\item описать структуру разрабатываемого программного обеспечения;
	\item определить средства программной реализации выбранных алгоритмов;
	\item реализовать разработанные алгоритмы;
	\item провести функциональное тестирование программного обеспечения;
	\item провести сравнительный анализ по времени реализованных алгоритмов и числу сравнений при поиске;
	\item подготовить отчет о выполненной лабораторной работе.
\end{itemize}
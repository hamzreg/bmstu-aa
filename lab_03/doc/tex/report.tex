\documentclass[a4paper,14pt, unknownkeysallowed]{extreport}

% Текст 
\usepackage[english,russian]{babel} % Языки

\usepackage{cmap} % Улучшенный поиск русских слов в полученном pdf-файле
\usepackage[T2A]{fontenc} % Поддержка русских букв
\usepackage[utf8]{inputenc} % Кодировка utf8

\usepackage[14pt]{extsizes} % Установка 14pt (без этого тоже работает)


% Поля
\usepackage{geometry}
\geometry{left=30mm}
\geometry{right=15mm}
\geometry{top=20mm}
\geometry{bottom=20mm}


% Заголовки
\usepackage{titlesec}

\titleformat{\section}{\normalsize\bfseries}
{\thesection}{1em}{}

\titlespacing*{\chapter}{0pt}{-30pt}{8pt}
\titlespacing*{\section}{\parindent}{*4}{*4}
\titlespacing*{\subsection}{\parindent}{*4}{*4}

\usepackage{titlesec}
\titleformat{\chapter}{\LARGE\bfseries}{\thechapter}{20pt}{\LARGE\bfseries}
\titleformat{\section}{\Large\bfseries}{\thesection}{20pt}{\Large\bfseries}


% Интервал
\usepackage{setspace}
\onehalfspacing % Полуторный интервал

% Абзацный отступ
\frenchspacing
\usepackage{indentfirst}


% Листинг кода [C++]:
\usepackage{listings}
\usepackage{xcolor}

\lstset{ %
	language=C++,   					% выбор языка для подсветки	
	basicstyle=\small\sffamily,			% размер и начертание шрифта для подсветки кода
	numbers=left,						% где поставить нумерацию строк (слева\справа)
	%numberstyle=,					% размер шрифта для номеров строк
	stepnumber=1,						% размер шага между двумя номерами строк
	numbersep=5pt,						% как далеко отстоят номера строк от подсвечиваемого кода
	frame=single,						% рисовать рамку вокруг кода
	tabsize=4,							% размер табуляции по умолчанию равен 4 пробелам
	captionpos=t,						% позиция заголовка вверху [t] или внизу [b]
	breaklines=true,					
	breakatwhitespace=true,				% переносить строки только если есть пробел
	escapeinside={\//*}{*)},				% если нужно добавить комментарии в коде
	backgroundcolor=\color{white},
}


% PDF
\usepackage[justification=centering]{caption} % Настройка подписей float объектов
\usepackage[unicode,pdftex]{hyperref} % Ссылки в pdf
\hypersetup{hidelinks}


% Подписи
\captionsetup{labelsep=endash} % Объект - описание

% Таблицы
\usepackage{threeparttable}

% Рисунки
\usepackage{caption}
\captionsetup[figure]{name={Рисунок}} % Рис. -> Рисунок

\usepackage{float}
\usepackage[justification=centering]{caption}
\usepackage{pgfplots}
\pgfplotsset{compat=1.9}
\usetikzlibrary{datavisualization}
\usetikzlibrary{datavisualization.formats.functions}
\usepackage{graphicx}
\newcommand{\img}[3] {
    \begin{figure}[h]
        \center{\includegraphics[height=#1]{assets/img/#2}}
        \caption{#3}
        \label{img:#2}
    \end{figure}
}
\newcommand{\boximg}[3] {
    \begin{figure}[h]
        \center{\fbox{\includegraphics[height=#1]{assets/img/#2}}}
        \caption{#3}
        \label{img:#2}
    \end{figure}
}


% Формулы
\usepackage{amsmath}
\newcommand{\code}[1]{\texttt{#1}} % Шрифт пишущей машинки в формулах

\begin{document}

\include{title}
\renewcommand{\contentsname}{Содержание} % Оглавление -> Содержание
\tableofcontents
\setcounter{page}{2}
\chapter*{Введение}
\addcontentsline{toc}{chapter}{Введение}

Одной из ключевых операций при обработке данных является поиск. Время работы алгоритмов поиска зависит от размера набора данных. Для представления данных используются различные структуры, которые удобно использовать в конкретных задачах. Одной из таких структур данных является словарь \cite{dict}. Следовательно, становится актуальной задача быстрого поиска в словаре.

Целью данной лабораторной работы является изучение алгоритмов поиска в словаре. Для достижения поставленной цели требуется выполнить следующие задачи:

\begin{itemize}
	\item изучить структуру данных - словарь;
	\item рассмотреть алгоритмы поиска в словаре: полный перебор, бинарный поиск и поиск сегментами;
	\item привести схемы изучаемых алгоритмов;
	\item описать используемые типы и структуры данных;
	\item описать структуру разрабатываемого программного обеспечения;
	\item определить средства программной реализации выбранных алгоритмов;
	\item реализовать разработанные алгоритмы;
	\item провести функциональное тестирование программного обеспечения;
	\item провести сравнительный анализ по времени реализованных алгоритмов и числу сравнений при поиске;
	\item подготовить отчет о выполненной лабораторной работе.
\end{itemize}
\chapter{Аналитическая часть}

В данном разделе будут описаны алгоритмы сортировки выбором, Шелла и гномьей сортировки.

\section{Сортировка выбором}

Сортировка выбором\cite{virt} состоит из следующих шагов:

\begin{itemize}
	\item выбирается элемент неотсортированной части последовательности с наименьшим значением;
	\item выбранный элемент меняется местами с элементом, стоящим на первой позиции в неотсортированной части. Обмен не нужен, если это и есть минимальный элемент;
	\item повтор шагов 1 и 2 до тех пор, пока не останется только наибольший элемент.
\end{itemize}

\section{Сортировка Шелла}

Сортировка Шелла\cite{virt} является усовершенствованием сортировки вставками. В сортировке вставками на каждом шаге, берут элемент входной последовательности и передают в готовую последовательность, вставляя его на подходящее место. Д. Л. Шелл предложил следующие шаги:

\begin{itemize}
	\item выбирается некоторое расстояние d между элементами последовательности;
	\item сравниваются и сортируются значения, стоящие друг от друга на расстоянии d;
	\item шаг 2 повторяется для меньших значений d, не равных 1;
	\item при d, равном 1, элементы упорядочиваются сортировкой вставками.
\end{itemize}

Приемлема любая последовательность для d, с условием, что последнее значение равно 1.

\section{Гномья сортировка}

Гномья сортировка \cite{gnome-sort} выполняет следующие действия:

\begin{itemize}
	\item сравниваются текущий и предыдущий элементы последовательности;
	\item если они расположены в необходимом порядке, то осуществляется переход к следующему элементу;
	\item иначе происходит обмен. Если предыдущий элемент не был первым, осуществляется переход на один элемент назад.
\end{itemize}

Шаги повторяются, пока возможен переход к следующему элементу.

\section*{Вывод}

Были рассмотрены следующие алгоритмы сортировки: выбором, Шелла и гномья. 
Для указанных алгоритмов необходимо получить теоретическую оценку и доказать её экспериментально.
\chapter{Конструкторская часть}

В данном разделе будут представлены схемы алгоритмов сортировки выбором, Шеллом и гномьей сортировки и вычислены трудоемкости указанных алгоритмов.

\section{Разработка алгоритмов}

На рисунках представлены схемы алгоритмов сортировки выбором, Шеллом и гномьей сортировки.

\begin{figure}[H]
	\begin{center}
		\includegraphics[scale=0.4]{img/selection_sort.png}
	\end{center}
	\captionsetup{justification=centering}
	\caption{Схема алгоритма сортировки выбором}
	\label{img:selection_sort}
\end{figure}

\begin{figure}[H]
	\begin{center}
		\includegraphics[scale=0.5]{img/shell_sort.png}
	\end{center}
	\captionsetup{justification=centering}
	\caption{Схема алгоритма сортировки Шелла}
	\label{img:shell_sort}
\end{figure}

\newpage

\begin{figure}[H]
	\begin{center}
		\includegraphics[scale=0.5]{img/gnome_sort.png}
	\end{center}
	\captionsetup{justification=centering}
	\caption{Схема алгоритма гномьей сортировки}
	\label{img:gnome_sort}
\end{figure}

\section{Модель вычислений для оценки трудоёмкости алгоритмов}

Для определения трудоемкости алгоритмов необходимо ввести модель вычислений:

\begin{enumerate}
	\item операции из списка (\ref{for:operations}) имеют трудоемкость равную 1;
	\begin{equation}
		\label{for:operations}
		+, -, /, *, \%, =, +=, -=, *=, /=, \%=, ==, !=, <, >, <=, >=, [], ++, {-}-
	\end{equation}
	\item трудоемкость оператора выбора \code{if условие then A else B} рассчитывается, как (\ref{for:if});
	\begin{equation}
		\label{for:if}
		f_{if} = f_{\text{условия}} +
		\begin{cases}
			f_A, & \text{если условие выполняется,}\\
			f_B, & \text{иначе.}
		\end{cases}
	\end{equation}
	\item трудоемкость цикла рассчитывается, как (\ref{for:cycle});
	\begin{equation}
		\label{for:cycle}
		f_{for} = f_{\text{инициализации}} + f_{\text{сравнения}} + N(f_{\text{тела}} + f_{\text{инкремент}} + f_{\text{сравнения}})
	\end{equation}
	\item трудоемкость вызова функции равна 0.
\end{enumerate}

\section{Трудоёмкость алгоритмов}

\section*{Вывод}
% схемы алгоритмов, типы и структуры данных, все, что сам придумал
\chapter{Технологическая часть}

В данном разделе будут указаны средства реализации, будут представлены листинги кода, а также функциональные тесты.

\section{Средства реализации}

Реализация данной лабораторной работы выполнялась при помощи языка программирования Python \cite{python}. Выбор ЯП обусловлен простотой синтаксиса, большим числом библиотек и эффективностью визуализации данных.

Замеры времени проводились при помощи функции process\_time из библиотеки time \cite{python-time}.

\section{Листинги кода}

Реализации алгоритмов нахождения расстояния Левенштейна и расстояния Дамерау-Левенштейна представлены в листингах \ref{lst:recursive}-\ref{lst:dl}.
\clearpage
\begin{center}
\captionsetup{justification=raggedright,singlelinecheck=off}
\begin{lstlisting}[label=lst:recursive,caption=Рекурсивный алгоритм нахождения расстояния Левенштейна]
def recursive(str1, str2):
    n = len(str1)
    m = len(str2)

    if n == 0 or m == 0:
        return abs(n - m)

    cost_replacement = Cost.replacement if str1[-1] != str2[-1] else 0

    deletion = recursive(str1[:-1], str2) + Cost.deletion
    insertion = recursive(str1, str2[:-1]) + Cost.insertion
    replacement = recursive(str1[:-1], str2[:-1]) + cost_replacement

    min_distance = min(deletion, insertion, replacement)

    return min_distance
\end{lstlisting} 
\end{center}

\begin{center}
\captionsetup{justification=raggedright,singlelinecheck=off}
\begin{lstlisting}[label=lst:matrix,caption=Матричный алгоритм нахождения расстояния Левенштейна]
def matrix_(str1, str2):
    n, m = len(str1), len(str2)
    matrix = create_matrix(n + 1, m + 1)

    for i in range(1, n + 1):
        for j in range(1, m + 1):

            cost_replacement = Cost.replacement if str1[-1] != str2[-1] else 0
            matrix[i][j] = min(matrix[i - 1][j] + Cost.deletion,
                               matrix[i][j - 1] + Cost.insertion,
                               matrix[i - 1][j - 1] + cost_replacement)

    return matrix[n][m]
\end{lstlisting}
\end{center}
\clearpage
\begin{center}
\captionsetup{justification=raggedright,singlelinecheck=off}
\begin{lstlisting}[label=lst:cache,caption=Алгоритм нахождения расстояния Левенштейна с использованием матрицы]
def recursive_(str1, str2, n, m, matrix):
    if matrix[n][m] != -1:
        return matrix[n][m]

    if n == 0:
        matrix[n][m] = m
        return matrix[n][m]
    
    if n > 0 and m == 0:
        matrix[n][m] = n
        return matrix[n][m]

    cost_replacement = Cost.replacement if str1[n - 1] != str2[m - 1] else 0

    deletion = recursive_(str1, str2, n - 1, m, matrix) + Cost.deletion
    insertion = recursive_(str1, str2, n, m - 1, matrix) + Cost.insertion
    replacement = recursive_(str1, str2, n - 1, m - 1, matrix) + cost_replacement
    
    matrix[n][m] = min(deletion,
                       insertion,
                       replacement)

    return matrix[n][m]


def recursive_with_cache(str1, str2):
    n, m = len(str1), len(str2)
    matrix = [[-1] * (m + 1) for _ in range(n + 1)]

    recursive_(str1, str2, n, m, matrix)

    return matrix[n][m]
\end{lstlisting}
\end{center}
\clearpage
\begin{center}
\captionsetup{justification=raggedright,singlelinecheck=off}
\begin{lstlisting}[label=lst:dl,caption=Рекурсивный алгоритм нахождения расстояния Дамерау-Левенштейна]
def damerau_levenshtein(str1, str2):
    n = len(str1)
    m = len(str2)

    if n == 0 or m == 0:
        return abs(n - m)

    cost_replacement = Cost.replacement if str1[-1] != str2[-1] else 0

    deletion = damerau_levenshtein(str1[:-1], str2) + Cost.deletion
    insertion = damerau_levenshtein(str1, str2[:-1]) + Cost.insertion
    replacement = damerau_levenshtein(str1[:-1], str2[:-1]) + cost_replacement

    if n > 1 and m > 1 and str1[-1] == str2[-2] and str1[-2] == str2[-1]:
        transposition = damerau_levenshtein(str1[:-2], str2[:-2]) + Cost.transposition
        return min(deletion, insertion, replacement, transposition)

    return min(deletion, insertion, replacement)
\end{lstlisting} 
\end{center}

\section{Функциональные тесты}

В таблице \ref{tbl:func_test} приведены функциональные тесты для функций, реализующих алгоритмы нахождения редакционного расстояния. Все тесты пройдены успешно.

\begin{table}[h]
	\begin{center}
        \begin{threeparttable}
        \captionsetup{justification=raggedright,singlelinecheck=off}
		\caption{\label{tbl:func_test} Функциональные тесты}
		\begin{tabular}{|c|c|c|c|c|}
			\hline
			& & & \multicolumn{2}{c|}{Ожидаемый результат} \\
			\hline
			№&Строка 1&Строка 2&Левенштейн&Дамерау-Левенштейн\\
			\hline
            1&"пустая строка"&"пустая строка"&0&0 \\
            \hline
            2&"пустая строка"&привет&6&6 \\
            \hline
            3&солнышко&"пустая строка"&8&8 \\
            \hline
            4&улыбка&улыбк&1&1 \\
			\hline
			5&рай&раааай&3&3 \\
			\hline
            6&аромат&игра&5&5 \\
			\hline
			7&душа&счастье&7&7 \\
			\hline
			8&закат&азкта&4&2 \\
			\hline
			9&мир&рим&2&2 \\
			\hline
			10&оскар&окр&2&2 \\
			\hline
		\end{tabular}
        \end{threeparttable}
	\end{center}
\end{table}


\section{Вывод}

Были реализованы функции алгоритмов поиска редакционного расстояния. Было проведено функциональное тестирование указанных функций.
% язык, среда, инструменты, средства для замера времени и памяти, листинги, тестирование
\chapter{Исследовательская часть}

В данном разделе будут приведены примеры работы программы, и будет проведен сравнительный анализ реализованных алгоритмов умножения матриц по затраченному процессорному времени.

\section{Технические характеристики}

Тестирование проводилось на устройстве со следующими техническими характеристиками:

\begin{itemize}
	\item операционная система: Ubuntu 20.04.1 Linux x86\_64 \cite{linux};
	\item память : 8 GiB;
	\item процессор: AMD® Ryzen™ 3 3200u @ 2.6 GHz \cite{amd}.
\end{itemize}

Тестирование проводилось на ноутбуке, включенном в сеть электропитания. Во время тестирования ноутбук был нагружен только встроенными приложениями окружения, а также непосредственно системой тестирования.

\clearpage

\section{Демонстрация работы программы}

На рисунке \ref{img:example} приведен пример работы программы.

\begin{figure}[H]
	\begin{center}
		\includegraphics[scale=0.3]{img/example.png}
	\end{center}
	\captionsetup{justification=centering}
	\caption{Пример работы программы}
	\label{img:example}
\end{figure}

\section{Время выполнения алгоритмов}

Функция process\_time из библиотеки time ЯП Python возвращает  процессорное время в секундах - значение типа float.

Для замера времени необходимо получить значение времени до начала выполнения алгоритма, затем после её окончания. Чтобы получить результат, необходимо вычесть из второго значения первое.

Замеры проводились для матриц стоимостей, заполненных случайным образом, размером от 2 до 10. Результаты измерения времени приведены в таблице \ref{tbl:time} (в с), а их графическое представление - на рисунке \ref{img:time}.

\begin{center}
\captionsetup{justification=raggedright,singlelinecheck=off}
\begin{longtable}[c]{|p{4cm}|p{4cm}|p{4cm}|}
\caption{Результаты замеров времени\label{tbl:time}}\\ \hline
   Число городов & Полный перебор & Муравьиный алгоритм \\ \hline
   2 &   0.000327 &   0.010621 \\ \hline
   3 &   0.000039 &   0.015454 \\ \hline
   4 &   0.000121 &   0.031155 \\ \hline
   5 &   0.000706 &   0.049366 \\ \hline
   6 &   0.003409 &   0.078147 \\ \hline
   7 &   0.024055 &   0.106820 \\ \hline
   8 &   0.223502 &   0.149278 \\ \hline
   9 &   2.248946 &   0.227690 \\ \hline
  10 &  22.530937 &   0.299644 \\ \hline
\end{longtable}
\end{center}

\begin{figure}[H]
	\begin{center}
		\includegraphics[scale=0.5]{img/time.png}
	\end{center}
	\captionsetup{justification=centering}
	\caption{Сравнение по времени алгоритмов задачи коммивояжера}
	\label{img:time}
\end{figure}

\section{Параметризация}

Целью проведения параметризации является определение таких комбинаций параметров, при которых муравьиный алгоритм даёт наилучшие результаты.

В результате автоматической параметризации будет получена таблицы со следующими столбцами:
\begin{itemize}
	\item коэффициент видимости $\alpha$ - изменяющийся параметр;
	\item коэффициент испарения феромона $\rho$ - изменяющийся параметр;
	\item число дней $days$ - изменяющийся параметр;
	\item эталонный результат $ideal$;
	\item разность полученного при данных параметрах значения и эталонного $mistake$.
\end{itemize}

\subsection{Класс данных 1}

Класс данных 1 представляет собой матрицу стоимостей в диапозоне от 1 до 5 для 8 городов:

\begin{equation}
    \label{eq:kd1}
	K_{1} = \begin{pmatrix}
		0 & 4 & 3 & 2 & 4 & 5 & 5 & 3 \\
		4 & 0 & 4 & 5 & 5 & 1 & 4 & 5 \\
		3 & 4 & 0 & 5 & 4 & 1 & 2 & 1 \\
		2 & 5 & 5 & 0 & 1 & 3 & 1 & 5 \\
		4 & 5 & 4 & 1 & 0 & 2 & 5 & 4 \\
		5 & 1 & 1 & 3 & 2 & 0 & 4 & 5 \\
		5 & 4 & 2 & 1 & 5 & 4 & 0 & 2 \\
		3 & 5 & 1 & 5 & 4 & 5 & 2 & 0 \\
	\end{pmatrix}
\end{equation}

Для данного класса данных приведена таблица с выборкой параметров, которые наилучшим образом решают поставленную задачу.

\begin{center}
    \captionsetup{justification=raggedright,singlelinecheck=off}
    \begin{longtable}[c]{|c|c|c|c|c|}
    \caption{Параметры для класса данных 1\label{tbl:table_kd1}}\\ \hline
        $\alpha$ & $\rho$ & Days & Result & Mistake \\ \hline
 0.1 &  0.9 &  100 &    15 &     0 \\
 0.1 &  0.9 &  200 &    15 &     0 \\
 0.1 &  0.9 &  300 &    15 &     0 \\
 0.1 &  0.9 &  400 &    15 &     0 \\
 0.1 &  0.9 &  500 &    15 &     0 \\
\hline
 0.2 &  0.8 &  100 &    15 &     0 \\
 0.2 &  0.8 &  200 &    15 &     0 \\
 0.2 &  0.8 &  300 &    15 &     0 \\
 0.2 &  0.8 &  400 &    15 &     0 \\
 0.2 &  0.8 &  500 &    15 &     0 \\
\hline
 0.3 &  0.7 &  100 &    15 &     0 \\
 0.3 &  0.7 &  200 &    15 &     0 \\
 0.3 &  0.7 &  300 &    15 &     0 \\
 0.3 &  0.7 &  400 &    15 &     0 \\
 0.3 &  0.7 &  500 &    15 &     0 \\
\hline
 0.4 &  0.6 &  100 &    15 &     0 \\
 0.4 &  0.6 &  200 &    15 &     0 \\
 0.4 &  0.6 &  300 &    15 &     0 \\
 0.4 &  0.6 &  400 &    15 &     0 \\
 0.4 &  0.6 &  500 &    15 &     0 \\
\hline
 0.5 &  0.5 &  100 &    15 &     0 \\
 0.5 &  0.5 &  200 &    15 &     0 \\
 0.5 &  0.5 &  300 &    15 &     0 \\
 0.5 &  0.5 &  400 &    15 &     0 \\
 0.5 &  0.5 &  500 &    15 &     0 \\
\hline
 0.6 &  0.4 &  100 &    15 &     0 \\
 0.6 &  0.4 &  200 &    15 &     0 \\
 0.6 &  0.4 &  300 &    15 &     0 \\
 0.6 &  0.4 &  400 &    15 &     0 \\
 0.6 &  0.4 &  500 &    15 &     0 \\
\hline
 0.7 &  0.3 &  100 &    15 &     0 \\
 0.7 &  0.3 &  200 &    15 &     0 \\
 0.7 &  0.3 &  300 &    15 &     0 \\
 0.7 &  0.3 &  400 &    15 &     0 \\
 0.7 &  0.3 &  500 &    15 &     0 \\
\hline
 0.8 &  0.2 &  100 &    15 &     0 \\
 0.8 &  0.2 &  200 &    15 &     0 \\
 0.8 &  0.2 &  300 &    15 &     0 \\
 0.8 &  0.2 &  400 &    15 &     0 \\
 0.8 &  0.2 &  500 &    15 &     0 \\
\hline
 0.9 &  0.1 &  100 &    15 &     0 \\
 0.9 &  0.1 &  200 &    15 &     0 \\
 0.9 &  0.1 &  300 &    15 &     0 \\
 0.9 &  0.1 &  400 &    15 &     0 \\
 0.9 &  0.1 &  500 &    15 &     0 \\
\hline
\end{longtable}
\end{center}

\subsection{Класс данных 2}

Класс данных 2 представляет собой матрицу стоимостей в диапозоне от 5000 до 9000 для 8 городов:

\begin{equation}
    \label{eq:kd2}
	K_{1} = \begin{pmatrix}
		0 & 5466 & 8308 & 8068 & 7284 & 5635 & 6055 & 8129 \\
		5466 & 0 & 8205 & 7384 & 6794 & 6048 & 6174 & 6306 \\
		8308 & 8205 & 0 & 5485 & 7872 & 7981 & 7868 & 6912 \\
		8068 & 7384 & 5485 & 0 & 7002 & 6683 & 7544 & 8278 \\
		7284 & 6794 & 7872 & 7002 & 0 & 5159 & 8240 & 5663 \\
		5635 & 6048 & 7981 & 6683 & 5159 & 0 & 8801 & 8844 \\
		6055 & 6174 & 7868 & 7544 & 8240 & 8801 & 0 & 5493 \\
		8129 & 6306 & 6912 & 8278 & 5663 & 8844 & 5493 & 0 \\
	\end{pmatrix}
\end{equation}

Для данного класса данных приведена таблица с выборкой параметров, которые наилучшим образом решают поставленную задачу.

\begin{center}
    \captionsetup{justification=raggedright,singlelinecheck=off}
    \begin{longtable}[c]{|c|c|c|c|c|}
    \caption{Параметры для класса данных 2\label{tbl:table_kd2}}\\ \hline
        $\alpha$ & $\rho$ & days & ideal & mistake \\ \hline
 0.1 &  0.7 &  300 & 47326 &     0 \\
 0.1 &  0.7 &  400 & 47326 &     0 \\
 0.1 &  0.7 &  500 & 47326 &     0 \\
 \hline
 0.2 &  0.5 &  300 & 47326 &     0 \\
 0.2 &  0.5 &  400 & 47326 &     0 \\
 0.2 &  0.5 &  500 & 47326 &     0 \\
 \hline
 0.3 &  0.3 &  300 & 47326 &     0 \\
 0.3 &  0.3 &  400 & 47326 &     0 \\
 0.3 &  0.3 &  500 & 47326 &     0 \\
\hline
 0.4 &  0.1 &  300 & 47326 &     0 \\
 0.4 &  0.1 &  400 & 47326 &     0 \\
 0.4 &  0.1 &  500 & 47326 &     0 \\
\hline
 0.5 &  0.6 &  300 & 47326 &     0 \\
 0.5 &  0.6 &  400 & 47326 &     0 \\
 0.5 &  0.6 &  500 & 47326 &     0 \\
 \hline
 0.6 &  0.8 &  300 & 47326 &     0 \\
 0.6 &  0.8 &  400 & 47326 &     0 \\
 0.6 &  0.8 &  500 & 47326 &     0 \\
\hline
 0.7 &  0.2 &  300 & 47326 &     0 \\
 0.7 &  0.2 &  400 & 47326 &     0 \\
 0.7 &  0.2 &  500 & 47326 &     0 \\
\hline
 0.8 &  0.6 &  300 & 47326 &     0 \\
 0.8 &  0.6 &  400 & 47326 &     0 \\
 0.8 &  0.6 &  500 & 47326 &     0 \\
\hline
 0.9 &  0.9 &  300 & 47326 &     0 \\
 0.9 &  0.9 &  400 & 47326 &     0 \\
 0.9 &  0.9 &  500 & 47326 &     0 \\
\hline
\end{longtable}
\end{center}

\section{Вывод}

В результате эксперимента было получено, что для 2 городов полный перебор работает быстрее муравьиного алгоритма в 32 раза. Начиная с 8 городов, муравьиный алгоритм работает быстрее полного перебора: в 75 раз быстрее для 10 городов. Таким образом, муравьиный алгоритм необходимо использовать при большом числе городов - от 8 и более.

Также в результате проведения параметризации было установлено, что для первого класса данных лучшие результаты муравьиный алгоритм дает на следующих значениях параметров:
\begin{itemize}
	\item $\alpha$ = 0.1, $\rho$ = 0.1-0.5, 0.7-0.9;
	\item $\alpha$ = 0.2, $\rho$ = 0.1-0.7, 0.9;
	\item $\alpha$ = 0.3, $\rho$ = 0.2-0.6, 0.9;
	\item $\alpha$ = 0.4, $\rho$ = 0.5-0.9;
	\item $\alpha$ = 0.5, $\rho$ = 0.1, 0.5-0.9;
	\item $\alpha$ = 0.7, $\rho$ = 0.4-0.8.
\end{itemize}

Можно сделать вывод о том, что данные значения параметров следует использовать для матриц стоимостей в диапазоне от 1 до 5.

Для второго класса данных лучшие результаты муравьиный алгоритм дает на следующих значениях параметров:
\begin{itemize}
	\item $\alpha$ = 0.1, $\rho$ = 0.1, 0.4, 0.7;
	\item $\alpha$ = 0.2, $\rho$ = 0.3, 0.7, 0.9;
	\item $\alpha$ = 0.3, $\rho$ = 0.2, 0.6, 0.9;
	\item $\alpha$ = 0.4, $\rho$ = 0.7, 0.8;
	\item $\alpha$ = 0.5, $\rho$ = 0.2, 0.6-0.9;
	\item $\alpha$ = 0.5, $\rho$ = 0.3, 0.7-0.9;
	\item $\alpha$ = 0.6, $\rho$ = 0.2, 0.6-0.9;
	\item $\alpha$ = 0.8, $\rho$ = 0.1, 0.6;
	\item $\alpha$ = 0.9, $\rho$ = 0.9.
\end{itemize}

Можно сделать вывод о том, что данные значения параметров следует использовать для матриц стоимостей в диапазоне от 5000 до 9000.

Из результатов параметризации видно, что погрешность результата уменьшается при большом числе дней и меньшем значении коэффициента видимости.
% графики: 10 точек на графике примерно, то есть берём массивы от 10 до 100 с шагом 10 и замеряем по каждому алгоритму время для 3 случаев (лучший, худший, произвольный). Таким образом, 9 графиков для 3 алгоритмов. Здесь же надо указать, как измеряли время (подсказка: нужно замерить одно и то же несколько раз и усреднить).
\chapter*{Заключение}
\addcontentsline{toc}{chapter}{Заключение}

В результате исследования было получено, что эвристический метод на базе муравьиного алгоритма следует использовать при большом количество городов - от 8 и более, так как для 2 городов мураьвиный алгоритм работает медленнее полного перебора в 32 раза, а для 10 городов - быстрее в 75 раз.

Лучшие значения муравьиный алгоритм показывает при меньших значениях коэффициента видимости и при большом числе дней.

Цель, поставленная перед началом работы, была достигнута. В ходе лабораторной работы были решены следующие задачи:

\begin{itemize}
	\item были изучена задача коммивояжера и методы ее решения;
	\item были разработаны полный перебор и муравьиный алгоритм;
	\item был проведен сравнительный анализ реализованных алгоритмов;
	\item был подготовлен отчет о выполненной лабораторной работе.
\end{itemize}

\addcontentsline{toc}{chapter}{Список литературы}
\bibliographystyle{utf8gost705u}
\renewcommand\bibname{Список литературы} % Литература -> Список литературы
\bibliography{biblio}

\end{document}
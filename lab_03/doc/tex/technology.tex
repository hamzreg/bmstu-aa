\chapter{Технологическая часть}

В данном разделе будут указаны требования к программному обеспечению и средства реализации, будут представлены листинги кода, а также функциональные тесты.

\section{Требования к ПО}

К программе представлен ряд требований:

\begin{itemize}
	\item на вход подается массив целых чисел;
	\item на выходе - отсортированный массив, поданный на вход.
\end{itemize}

\section{Средства реализации}

Реализация данной лабораторной работы выполнялась при помощи языка программирования Python \cite{python}. Выбор ЯП обусловлен простотой синтаксиса, большим числом библиотек и эффективностью визуализации данных.

Замеры времени проводились при помощи функции process\_time из библиотеки time \cite{python-time}.

\section{Сведения о модулях программы}

Программа состоит из следующих модулей:

\begin{itemize}
	\item main.py - главный файл программы, предоставляющий пользователю меню для выполнения основных функций;
	\item sort.py - файл, содержащий функции сортировок массива;
	\item arr.py - файл, содержащий функции создания массива различного типа и работы с массивом;
	\item time\_test.py - файл, содержащий функции замеров времени работы сортировок;
	\item graph\_result.py - файл, содержащий функции визуализации временных характеристик алгоритмов сортировок.
\end{itemize}

\section{Листинги кода}

Реализации алгоритмов сортировок выбором, Шелла и гномьей представлены на листингах \ref{lst:selection_sort}, \ref{lst:shell_sort}, \ref{lst:gnome_sort}.

\begin{center}
\captionsetup{justification=raggedright,singlelinecheck=off}
\begin{lstlisting}[label=lst:selection_sort,caption=Алгоритм сортировки выбором]
def selection_sort(arr, size):
    for i in range(size - 1):
        min_element = arr[i]
        min_index = i

        for j in range(i + 1, size):
            if arr[j] < min_element:
                min_element = arr[j]
                min_index = j
        
        arr[i], arr[min_index] = arr[min_index], arr[i]

	return arr
\end{lstlisting} 
\end{center}

\begin{center}
\captionsetup{justification=raggedright,singlelinecheck=off}
\begin{lstlisting}[label=lst:shell_sort,caption=Алгоритм сортировки Шелла]
def shell_sort(arr, size):
    d = size // 2

    while d > 0:
        for i in range(0, size):
            for j in range(i + d, size, d):
                if arr[i] > arr[j]:
                    arr[i], arr[j] = arr[j], arr[i]
        d //= 2

	return arr
\end{lstlisting}
\end{center}

\begin{center}
\captionsetup{justification=raggedright,singlelinecheck=off}
\begin{lstlisting}[label=lst:gnome_sort,caption=Алгоритм гномьей сортировки]
def gnome_sort(arr, size):
    i = 1

    while i < size:
        if arr[i - 1] <= arr[i]:
            i += 1
        else:
            arr[i - 1], arr[i] = arr[i], arr[i - 1]

            if i > 1:
                i -= 1

	return arr
\end{lstlisting}
\end{center}

\section{Функциональные тесты}

В таблице \ref{tbl:func-tests} приведены функциональные тесты для функций, реализующих алгоритмы сортировок. Все тесты пройдены успешно.

\begin{table}[h]
	\begin{center}
		\caption{\label{tbl:func-tests} Функциональные тесты}
		\begin{tabular}{|c|c|c|}
			\hline
			Входной массив & Ожидаемый результат & Результат \\ 
			\hline
			$[1, 2, 3, 4, 5, 6]$ & $[1, 2, 3, 4, 5, 6]$  & $[1, 2, 3, 4, 5, 6]$\\
			$[5, 4, 3, 2, 1]$  & $[1, 2, 3, 4, 5]$ & $[1, 2, 3, 4, 5]$\\
			$[0, -5, 3, 7, -8]$  & $[-8, -5, 0, 3, 7]$  & $[-8, -5, 0, 3, 7]$\\
			$[17]$  & $[17]$  & $[17]$\\
			$[]$  & $[]$  & $[]$\\
			\hline
		\end{tabular}
	\end{center}
\end{table}

\section*{Вывод}

Были реализованы функции алгоритмов сортировки выбором, Шелла и гномьей. Было проведено функциональное тестирование указанных функций.
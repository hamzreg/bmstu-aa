\chapter{Аналитическая часть}

В данном разделе будут описаны алгоритмы сортировки выбором, Шелла и гномьей сортировки.

\section{Сортировка выбором}

Сортировка выбором\cite{virt} состоит из следующих шагов:

\begin{enumerate}
	\item Выбирается элемент неотсортированной части последовательности с наименьшим значением;
	\item Выбранный элемент меняется местами с элементом, стоящим на первой позиции в неотсортированной части. Обмен не нужен, если это и есть минимальный элемент;
	\item Повтор шагов 1 и 2 до тех пор, пока не останется только наибольший элемент.
\end{enumerate}

\section{Сортировка Шелла}

Сортировка Шелла\cite{virt} является усовершенствованием сортировки вставками. В сортировке вставками на каждом шаге, берут элемент входной последовательности и передают в готовую последовательность, вставляя его на подходящее место. Д. Л. Шелл предложил следующие шаги:

\begin{enumerate}
	\item Выбирается некоторое расстояние d между элементами последовательности;
	\item Сравниваются и сортируются значения, стоящие друг от друга на расстоянии d;
	\item Шаг 2 повторяется для меньших значений d, не равных 1;
	\item При d, равном 1, элементы упорядочиваются сортировкой вставками.
\end{enumerate}

Приемлема любая последовательность для d, с условием, что последнее значение равно 1.

\section{Гномья сортировка}

Гномья сортировка выполняет следующие действия:

\begin{enumerate}
	\item Сравниваются текущий и предыдущий элементы последовательности;
	\item Если они расположены в необходимом порядке, то осуществляется переход к следующему элементу.
	\item Иначе происходит обмен. Если предыдущий элемент не был первым, осуществляется переход на один элемент назад.
\end{enumerate}

Шаги повторяются, пока возможен переход к следующему элементу.

\section*{Вывод}

Были рассмотрены следующие алгоритмы сортировки: выбором, Шелла и гномья. Для указанных алгоритмов необходимо получить теоретическую оценку и доказать её экспериментально.
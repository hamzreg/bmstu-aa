\chapter*{Введение}
\addcontentsline{toc}{chapter}{Введение}

При решении различных задач встает необходимость работы с упорядоченным набором данных. Например, при поиске элемента в заданном множестве. Для упорядочивания последовательности значений используется сортировка.

Сортировка - процесс перегруппировки заданного множества объектов в некотором определенном порядке \cite{virt}. Для реализации этого процесса разрабатываются алгоритмы сортировки. Такие алгоритмы состоят из трех основных шагов: 

\begin{itemize}
	\item сравнение элементов, задающее их порядок;
	\item обмен элементов в паре;
	\item сортирующий алгоритм, осуществляющий предыдущие два шага до полного упорядочивания.
\end{itemize}

Эффективность алгоритма зависит от скорости работы этого алгоритма. Скорость работы алгоритма сортировки определяется функциональной зависимостью среднего времени сортировки последовательностей элементов
данных, определенной длины, от этой длины.

Существует большое количество алгоритмов сортировки. Все они решают одну и ту же задачу, причем некоторые алгоритмы имеют преимущества перед другими. Поэтому существует необходимость сравнительного анализа алгоритмов сортировки. 

Цель работы - получить навык сравнительного анализа алгоритмов сортировки.

Для решения поставленной цели требуется решить следующие задачи:

\begin{itemize}
	\item изучить три алгоритма сортировки: выбором, Шелла и гномью;
	\item разработать и реализовать указанные алгоритмы;
	\item протестировать реализацию рассматриваемых алгоритмов;
	\item провести сравнительный анализ реализованных алгоритмов по затраченному процессорному времени, по трудоемкости и по памяти.
\end{itemize}
\chapter*{Введение}
\addcontentsline{toc}{chapter}{Введение}

Существуют задачи, в которых разные алгоритмы обрабатывают один и тот же набор данных друг за другом. При этом, может стоять задача обработки большого объема данных. Для ускорения решения таких вычислительных задач используется конвейерная обработка.

Идея конвейерной обработки \cite{idea} заключается в выделении отдельных этапов выполнения общей операции. Каждый этап, выполнив свою работу, передает результат следующему, одновременно принимая новую порцию входных данных. Отдельный этап называют лентой. При таком способе организации вычислений увеличивается скорость обработки за счет совмещения прежде разнесенных во времени операций.

В многопоточном программировании конвейерная обработка реализуется следующим образом: под каждую ленту конвейера выделяется отдельный поток. Выделенные потоки работают асинхронно.

Целью данной лабораторной работы является организация асинхронного взаимодействия потоков на примере конвейерной
обработки данных. Для достижения поставленной цели требуется выполнить следующие задачи:

\begin{itemize}
	\item описать последовательный вариант алгоритма задачи;
	\item выделить алгоритмы, выполняемые на отдельных лентах конвейера;
	\item описать конвейерный вариант алгоритма задачи;
	\item привести схемы рассмотренных вариантов алгоритма;
	\item описать используемые типы и структуры данных;
	\item описать структуру разрабатываемого программного обеспечения;
	\item определить средства программной реализации изученных вариантов алгоритма;
	\item реализовать разработанные варианты алгоритма;
	\item провести функциональное тестирование программного обеспечения;
	\item провести сравнительный анализ по времени последовательной и конвейерной реализаций;
	\item подготовить отчет о выполненной лабораторной работе.
\end{itemize}
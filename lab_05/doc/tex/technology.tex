\chapter{Технологическая часть}

В данном разделе будут указаны средства реализации, сведения о модулях программы, будут представлены листинги кода, а также функциональные тесты.

\section{Средства реализации}

Реализация данной лабораторной работы выполнялась при помощи языка программирования $C++$ \cite{c++}. Выбор ЯП обусловлен наличием инструментов для реализации принципов многопоточного программирования.

Замеры времени проводились при помощи функции $std::chrono::system\_clock::now(...)$ из библиотеки $chrono$ \cite{time}.

Реализация графического представления замеров времени производилась при помощи языка программирования $Python$ \cite{python}, так как данный ЯП представляет простую в использовании графическую библиотеку.

\section{Сведения о модулях программы}

Программа состоит из следующих модулей:

\begin{itemize}
	\item main.cpp - главный файл программы, предоставляющий пользователю меню для выполнения основных функций;
	\item str.cpp и str.h - файлы, содержащие функции работы со строкой;
	\item conveyor.cpp и conveyor.h - файлы, содержащие функции обработки данных;
	\item time\_test.cpp и time\_test.h - файлы, содержащие функции замеров времени работы указанных алгоритмов;
	\item constants.h - файл, содержащий все необходимые константы;
	\item graph\_result.py - файл, содержащий функции визуализации временных характеристик описанных алгоритмов.
\end{itemize}

\section{Листинги кода}

В листингах \ref{lst:linear}-\ref{lst:parallel} представлены алгоритмы последовательной и конвейерной обработки строк. В листингах \ref{lst:first}-\ref{lst:third} представлены алгоритмы работы трех лент. В листингах \ref{lst:spaces}-\ref{lst:palindrome} представлены алгоритмы работы со строками.
\clearpage
\begin{center}
    \captionsetup{justification=raggedright,singlelinecheck=off}
    \begin{lstlisting}[label=lst:linear,caption=Последовательная обработка]
void linear_mode(const int count, const int len,
                 const int is_palindrome)
{
    std::queue<std::string> q1;
    std::queue<std::string> q2;
    std::queue<std::string> q3;

    for ( int i =0; i < count; ++i )
    {
        std::string new_str;
        create_str(new_str, len, is_palindrome);

        q1.push(new_str);
    }

    cur_time = 0;

    for ( int i = 0; i < count; ++i )
    {
        std::string str = q1.front();
        linear_log(str, 1, i + 1);
        q2.push(str);
        q1.pop();

        str = q2.front();
        linear_log(str, 2, i + 1);
        q3.push(str);
        q2.pop();

        str = q3.front();
        linear_log(str, 3, i + 1);
        q3.pop();
    }
}
\end{lstlisting}
\end{center}
\clearpage
\begin{center}
    \captionsetup{justification=raggedright,singlelinecheck=off}
    \begin{lstlisting}[label=lst:parallel,caption=Конвейерная обработка]
void parallel_mode(const int count, const int len,
                 const int is_palindrome)
{
    std::queue<std::string> q1;
    std::queue<std::string> q2;
    std::queue<std::string> q3;

    for ( int i = 0; i < count; ++i )
    {
        std::string str;
        create_str(str, len, is_palindrome);

        q1.push(str);
    }
    
    for ( int i = 0; i < count + 1; ++i )
    {
        cur_time1.push_back(0);
        cur_time2.push_back(0);
        cur_time3.push_back(0);
    }

    std::vector<std::thread> threads(COUNT_THREADS);

    threads[0] = std::thread(first_belt, std::ref(q1), std::ref(q2));
    threads[1] = std::thread(second_belt, std::ref(q1), std::ref(q2), std::ref(q3));
    threads[2] = std::thread(third_belt, std::ref(q1), std::ref(q2), std::ref(q3));

    for ( int i = 0; i < COUNT_THREADS; ++i )
    {
        threads[i].join();
    }
}
\end{lstlisting}
\end{center}
\clearpage
\begin{center}
    \captionsetup{justification=raggedright,singlelinecheck=off}
    \begin{lstlisting}[label=lst:first,caption=Алгоритм работы первой ленты]
void first_belt(std::queue<std::string>& q1, std::queue<std::string>& q2)
{
    int task = 0;
    std::mutex m;

    while ( !q1.empty())
    {
        m.lock();
        std::string str = q1.front();
        m.unlock();

        parallel_log(str, 1, ++task);

        m.lock();
        q2.push(str);
        q1.pop();
        m.unlock();
    }
}
\end{lstlisting}
\end{center}
\clearpage
\begin{center}
    \captionsetup{justification=raggedright,singlelinecheck=off}
    \begin{lstlisting}[label=lst:second,caption=Алгоритм работы второй ленты]
void second_belt(std::queue<std::string>& q1, std::queue<std::string>& q2, std::queue<std::string>& q3)
{
    int task = 0;
    std::mutex m;

    do
    {
        if ( !q2.empty() )
        {
            m.lock();
            std::string str = q2.front();
            m.unlock();

            parallel_log(str, 2, ++task);

            m.lock();
            q3.push(str);
            q2.pop();
            m.unlock();
        }
    } while ( !q1.empty() || !q2.empty() );
}
\end{lstlisting}
\end{center}
\clearpage
\begin{center}
    \captionsetup{justification=raggedright,singlelinecheck=off}
    \begin{lstlisting}[label=lst:third,caption=Алгоритм работы третьей ленты]
void third_belt(std::queue<std::string>& q1, std::queue<std::string>& q2, std::queue<std::string>& q3)
{
    int task = 0;
    std::mutex m;

    do
    {
        if ( !q3.empty() )
        {
            m.lock();
            std::string str = q3.front();
            m.unlock();

            parallel_log(str, 3, ++task);

            m.lock();
            q3.pop();
            m.unlock();
        }
    } while ( !q1.empty() || !q2.empty() || !q3.empty() );
}
\end{lstlisting}
\end{center}

\begin{center}
    \captionsetup{justification=raggedright,singlelinecheck=off}
    \begin{lstlisting}[label=lst:spaces,caption=Алгоритм удаления пробелов]
void remove_spaces(std::string& str)
{
    const char space = ' ';
    int len = str.length();

    for ( int i = 0; i < len; ++i )
    {
        if ( str[i] == space )
        {
            str.erase(i, 1);
            i--;
        }
    }
}
\end{lstlisting}
\end{center}

\begin{center}
    \captionsetup{justification=raggedright,singlelinecheck=off}
    \begin{lstlisting}[label=lst:register,caption=Алгоритм перевода в другой регистр]
void change_register(std::string& str)
{
    int len = str.length();

    for ( int i = 0; i < len; ++i )
    {
        if ( str[i] >= 'A' && str[i] <= 'Z' )
        {
            str[i] += DIFF;
        }
        else if ( str[i] >= 'a' && str[i] <= 'z' )
        {
            str[i] -= DIFF;
        }
    }
}
\end{lstlisting}
\end{center}

\begin{center}
    \captionsetup{justification=raggedright,singlelinecheck=off}
    \begin{lstlisting}[label=lst:palindrome,caption=Алгоритм определения палиндрома]
int check_palindrome(std::string& str)
{
    int result = PALINDROME;
    int len = str.length();

    for ( int i = 0; i < len / 2; ++i )
    {
        if ( str[i] != str[len - i - 1] )
        {
            result = NOT_PALINDROME;
            break;
        }
    }

    return result;
}
\end{lstlisting}
\end{center}

\section{Функциональные тесты}

В таблице \ref{tbl:func_test} приведены функциональные тесты для функций, реализующих строковые алгоритмы. Все тесты пройдены успешно.

\begin{table}[h]
	\begin{center}
		\begin{threeparttable}
		\captionsetup{justification=raggedright,singlelinecheck=off}
		\caption{\label{tbl:func_test} Функциональные тесты}
		\begin{tabular}{|c|c|c|c|}
			\hline
			Обработка & Число строк & Длина строк & Ожидаемый результат \\ 
			\hline
			Последовательная & 0 & 100 & Сообщение об ошибке \\
			\hline
			Конвейерная & 50 & -2 & Сообщение об ошибке \\
			\hline
			Последовательная & 50 & 1000 & Лог обработки \\
			\hline
			Конвейерная & 60 & 400 & Лог обработки \\
			\hline
		\end{tabular}
		\end{threeparttable}
	\end{center}
	
\end{table}

\section{Вывод}

Были реализованы функции последовательной и конвейерной обработки набора строк. Было проведено функциональное тестирование указанных функций.
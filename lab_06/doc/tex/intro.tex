\chapter*{Введение}
\addcontentsline{toc}{chapter}{Введение}

Современные средства навигации, организация логистики, конвейерного производства, анализ эффективности финансовых инструментов строятся на алгоритмах решения задачи поиска оптимального решения по выбранному параметру в сложной системе.
Данную задачу высокой вычислительной сложности называют задачей коммивояжера \cite{task}.

Задачи высокой вычислительной сложности могут быть решены при помощи полного перебора вариантов и эвристических алгоритмов \cite{evr}. Смысл понятия ''эвристический алгоритм'' состоит в том, что в этом случае алгоритм не вытекает из строгих положений теории, а в значительной степени основан на интуиции и опыте. Такие методы могут давать удовлетворительные результаты при вероятностных параметрах. Алгоритмы, основанные на использовании эвристических алгоритмов, не всегда приводят к оптимальным решениям. Однако для их применения на практике достаточно, чтобы ошибка прогнозирования не превышала допустимого значения, а этого можно добиться, например, подбором более информативных параметров.

Целью данной лабораторной является сравнительный анализ метода полного перебора и эвристического метода на базе муравьиного алгоритма. Для достижения поставленной цели требуется выполнить следующие задачи:

\begin{itemize}
	\item изучить задачу коммивояжера;
	\item рассмотреть методы ее решения: полный перебор вариантов и муравьиный алгоритм;
	\item привести схемы изучаемых алгоритмов;
	\item описать используемые типы и структуры данных;
	\item описать структуру разрабатываемого программного обеспечения;
	\item определить средства программной реализации выбранных алгоритмов;
	\item реализовать разработанные алгоритмы;
	\item провести функциональное тестирование программного обеспечения;
	\item провести сравнительный анализ по времени реализованных алгоритмов;
	\item провести параметризацию муравьиного алгоритма;
	\item подготовить отчет о выполненной лабораторной работе.
\end{itemize}
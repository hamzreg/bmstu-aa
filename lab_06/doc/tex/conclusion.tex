\chapter*{Заключение}
\addcontentsline{toc}{chapter}{Заключение}

В результате исследования было получено, что эвристический метод на базе муравьиного алгоритма следует использовать при большом количество городов - от 8 и более, так как для 2 городов мураьвиный алгоритм работает медленнее полного перебора в 32 раза, а для 10 городов - быстрее в 75 раз.

Лучшие значения муравьиный алгоритм показывает при меньших значениях коэффициента видимости и при большом числе дней.

Цель, поставленная перед началом работы, была достигнута. В ходе лабораторной работы были решены следующие задачи:

\begin{itemize}
	\item были изучена задача коммивояжера и методы ее решения;
	\item были разработаны полный перебор и муравьиный алгоритм;
	\item был проведен сравнительный анализ реализованных алгоритмов;
	\item был подготовлен отчет о выполненной лабораторной работе.
\end{itemize}

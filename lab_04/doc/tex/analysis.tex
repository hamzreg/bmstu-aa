\chapter{Аналитическая часть}

В данном разделе будут описаны основы многопоточности, способ представления графа и алгоритм Флойда поиска кратчайших расстояний между всеми парами вершин графа.

\section{Многопоточность}

\section{Представление графа}

\section{Алгоритм Флойда}

\section{Вывод}

Были изучены основы распараллеливания алгоритмов, способ представления графа при помощи матрицы смежности и алгоритм Флойда поиска кратчайших путей между всеми парами вершин графа.

Программе, реализующей данный алгоритм, на вход будет подаваться матрица смежности, которая задает граф. Выходными данными такой программы должна быть матрица смежности кратчайших путей между всеми парами вершин графа. Программа должна работать в рамках следующих ограничений:

\begin{itemize}
	\item веса ребер графа - целые неотрицательные числа;
	\item при отсутствии пути между вершинами обозначается -1;
	\item должно быть выдано сообщение об ошибке при вводе пустой матрицы смежности или при вводе недопустимого веса ребра графа.
\end{itemize}

Пользователь должен иметь возможность выбора построения алгоритма с распараллеливанием и без него. В случае параллельного алгоритма пользователь должен иметь возможность ввода числа потоков. Также должны быть реализованы сравнение алгоритмов по времени работы в зависимости от числа потоков и в зависимости от порядка графа с выводом результатов на экран и получение графического представления результатов сравнения. Результат данных действий пользователь должен получать при помощи меню.
\chapter*{Введение}
\addcontentsline{toc}{chapter}{Введение}

Одной из задач программирования является ускорение решения вычислительных задач. Один из способов ее решения - использование параллельных вычислений.

В последовательном алгоритме решения какой-либо задачи есть операции, которые может выполнять только один процесс, например, операции ввода и вывода. Кроме того, в алгоритме могут быть операции, которые могут выполняться параллельно разными процессами. Алгоритм, операции которого могут быть выполнены разными процессами параллельно, называют параллельным. Каждый процесс состоит из одного или нескольких потоков. Свойство, состоящее в разделении процесса на потоки, выполняющие задачи параллельно, называют многопоточностью.

Примером вычислительных задач являются алгоритмы обработки графов. Графом называют конечное множество вершин и множество ребер. Каждому ребру сопоставлены две вершины - концы ребра. Число вершин графа называют порядком. Для распараллеливания может быть рассмотрена задача поиска кратчайших путей между всеми парами вершин графа. Данная задача решается при помощи алгоритма Флойда.

Целью данной лабораторной работы является изучение многопоточности на основе алгоритма Флойда поиска кратчайших расстояний между всеми парами вершин графа. Для достижения поставленной цели требуется выполнить следующие задачи:

\begin{itemize}
	\item изучить основы многопоточности;
	\item изучить способ представления графа;
	\item изучить алгоритм поиска кратчайших расстояний между всеми парами вершин графа;
	\item привести схемы изучаемого алгоритма;
	\item описать используемые типы и структуры данных;
	\item описать структуру разрабатываемого программного обеспечения;
	\item определить средства программной реализации выбранного алгоритма;
	\item реализовать разработанный алгоритм;
	\item провести функциональное тестирование программного обеспечения;
	\item провести сравнительный анализ по времени реализованного алгоритма;
	\item подготовить отчет о выполненной лабораторной работе.
\end{itemize}
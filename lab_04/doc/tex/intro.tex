\chapter*{Введение}
\addcontentsline{toc}{chapter}{Введение}

Одной из задач программирования является ускорение решения вычислительных задач. Один из способов ее решения - использование параллельных вычислений.

В последовательном алгоритме решения какой-либо задачи есть операции, которые может выполнять только один процесс, например, операции ввода и вывода. Кроме того, в алгоритме могут быть операции, которые могут выполняться параллельно разными процессами. Способность центрального процессора или одного ядра в многоядерном процессоре одновременно выполнять несколько процессов или потоков, соответствующим образом поддерживаемых операционной системой, называют многопоточностью \cite{multithreading}.

Процессом является программа в ходе своего выполнения. Каждый процесс состоит из одного или нескольких потоков - исполняемых сущностей, которые выполняют задачи, стоящие перед исполняемым приложением. После окончания выполнения всех потоков завершается процесс.

Современные процессоры могут выполнять две задачи на одном ядре при помощи дополнительного виртуального ядра. Такие процессоры называются многоядерными. Каждое ядро может выполнять только один поток за единицу времени. Если потоки выполняются последовательно, то их выполняет только одно ядро процессора, другие ядра не задействуются. Если независимые вычислительные задачи будут выполняться несколькими потоками параллельно, то будет задействовано несколько ядер процессора и решение задач ускорится.

Для распараллеливания может быть рассмотрена задача поиска кратчайших путей между всеми парами вершин графа. Данная задача решается при помощи алгоритма Флойда.

Целью данной лабораторной работы является изучение многопоточности на основе алгоритма Флойда поиска кратчайших расстояний между всеми парами вершин графа. Для достижения поставленной цели требуется выполнить следующие задачи:

\begin{itemize}
	\item изучить основы теории графов;
	\item выбрать способ представления графа;
	\item изучить последовательный и параллельный варианты алгоритма поиска кратчайших расстояний между всеми парами вершин графа;
	\item привести схемы изучаемого алгоритма;
	\item описать используемые типы и структуры данных;
	\item описать структуру разрабатываемого программного обеспечения;
	\item определить средства программной реализации выбранного алгоритма;
	\item реализовать разработанный алгоритм;
	\item провести функциональное тестирование программного обеспечения;
	\item провести сравнительный анализ по времени реализованного алгоритма;
	\item подготовить отчет о выполненной лабораторной работе.
\end{itemize}
\chapter{Технологическая часть}

В данном разделе будут указаны средства реализации, будут представлены листинги кода, а также функциональные тесты.

\section{Средства реализации}

Реализация данной лабораторной работы выполнялась при помощи языка программирования Python \cite{python}. Выбор ЯП обусловлен простотой синтаксиса, большим числом библиотек и эффективностью визуализации данных.

Замеры времени проводились при помощи функции process\_time из библиотеки time \cite{python-time}.

\section{Сведения о модулях программы}

Программа состоит из следующих модулей:

\begin{itemize}
	\item main.py - главный файл программы, предоставляющий пользователю меню для выполнения основных функций;
	\item matrix.py - файл, содержащий функции работы с матрицами;
	\item time\_test.py - файл, содержащий функции замеров времени работы указанных алгоритмов;
	\item graph\_result.py - файл, содержащий функции визуализации временных характеристик описанных алгоритмов.
\end{itemize}

\section{Листинги кода}



\section{Функциональные тесты}

В таблице \ref{tbl:func_test} приведены функциональные тесты для функций, реализующих алгоритмы умножения матриц. Все тесты пройдены успешно.

\begin{table}[h]
	\begin{center}
		\begin{threeparttable}
		\captionsetup{justification=raggedright,singlelinecheck=off}
		\caption{\label{tbl:func_test} Функциональные тесты}
		\begin{tabular}{|c@{\hspace{7mm}}|c@{\hspace{7mm}}|c@{\hspace{7mm}}|c@{\hspace{7mm}}|c@{\hspace{7mm}}|c@{\hspace{7mm}}|}
			\hline
			Матрица A & Матрица B & Ожидаемый результат \\ 
			\hline
			$\begin{pmatrix}
				&
			\end{pmatrix}$ &
			$\begin{pmatrix}
				&
			\end{pmatrix}$ &
			Сообщение об ошибке \\ \hline

			$\begin{pmatrix}
				&
			\end{pmatrix}$ &
			$\begin{pmatrix}
				1 & 2\\
				3 & 4\\
				5 & 6
			\end{pmatrix}$ &
			Сообщение об ошибке \\ \hline

			$\begin{pmatrix}
				1 & 0 & 1
			\end{pmatrix}$ &
			$\begin{pmatrix}
				-1 & 0 & -1
			\end{pmatrix}$ &
			Сообщение об ошибке \\ \hline

			$\begin{pmatrix}
				1 & 2 & 3 \\
				4 & 5 & 6 \\
				7 & 8 & 9
			\end{pmatrix}$ &
			$\begin{pmatrix}
				-1 & -2 & -3 \\
				-4 & -5 & -6 \\
				-7 & -8 & -9
			\end{pmatrix}$ &
			$\begin{pmatrix}
				-30 & -36 & -42 \\
				-66 & -81 & -96 \\
				-102 & -126 & -150
			\end{pmatrix}$ \\ \hline

			$\begin{pmatrix}
				1 & 2 & 3
			\end{pmatrix}$ &
			$\begin{pmatrix}
				1 \\
				2 \\
				3
			\end{pmatrix}$ &
			$\begin{pmatrix}
				14 \\
			\end{pmatrix}$ \\ \hline

		\end{tabular}
		\end{threeparttable}
	\end{center}
	
\end{table}

\section{Вывод}

Были реализованы функции алгоритмов умножения матриц. Было проведено функциональное тестирование указанных функций.
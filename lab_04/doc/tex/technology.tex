\chapter{Технологическая часть}

В данном разделе будут указаны средства реализации, сведения о модулях программы, будут представлены листинги кода, а также функциональные тесты.

\section{Средства реализации}

Реализация данной лабораторной работы выполнялась при помощи языка программирования $C++$ \cite{c++}. Выбор ЯП обусловлен наличием инструментов для реализации принципов многопоточного программирования.

Замеры времени проводились при помощи функции $std::chrono::system\_clock::now(...)$ из библиотеки $chrono$ \cite{time}.

Реализация графического представления замеров времени производилась при помощи языка программирования $Python$ \cite{python}, так как данный ЯП представляет простую в использовании графическую библиотеку.

\section{Сведения о модулях программы}

Программа состоит из следующих модулей:

\begin{itemize}
	\item main.cpp - главный файл программы, предоставляющий пользователю меню для выполнения основных функций;
	\item graph.cpp и graph\_t.h - файлы, содержащие функции работы с графом;
	\item time\_test.cpp и time\_test.h - файлы, содержащие функции замеров времени работы указанных алгоритмов;
	\item constants.h - файл, содержащий все необходимые константы;
	\item graph\_result.py - файл, содержащий функции визуализации временных характеристик описанных алгоритмов.
\end{itemize}

\section{Листинги кода}

В листинге \ref{lst:min} представлен алгоритм нахождения кратчайшего пути между двумя вершинами. В листингах \ref{lst:floyd}-\ref{lst:parallel} представлены последовательный и параллельный алгоритмы Флойда.

\begin{center}
    \captionsetup{justification=raggedright,singlelinecheck=off}
    \begin{lstlisting}[label=lst:min,caption=Алгоритм нахождения кратчайшего пути между двумя вершинами]
void find_min_way(int **matrix, int i, int j, int k)
{
    if (i != j && matrix[i][k] != NO_WAY && matrix[k][j] != NO_WAY)
    {
        if (matrix[i][j] == NO_WAY)
            matrix[i][j] = matrix[i][k] + matrix[k][j];
        else
            matrix[i][j] = min(matrix[i][j], matrix[i][k] + matrix[k][j]);
    }
}
\end{lstlisting}
\end{center}

\begin{center}
    \captionsetup{justification=raggedright,singlelinecheck=off}
    \begin{lstlisting}[label=lst:floyd,caption=Последовательный алгоритм Флойда]
void floyd(graph_t* graph)
{
    for (int k = 0; k < graph->order; k++)
    {
        for (int i = 0; i < graph->order; i++)
        {
            for (int j = 0; j < graph->order; j++)
            {
                find_min_way(graph->matrix, i, j, k);
            }
        }
    }
}
\end{lstlisting}
\end{center}

\begin{center}
    \captionsetup{justification=raggedright,singlelinecheck=off}
    \begin{lstlisting}[label=lst:parallel,caption=Параллельный алгоритм Флойда]
void parallel_floyd(graph_t* graph, int count_threads, int thread_index)
{
    int step = graph->order / count_threads;
    int start = thread_index * step;

    if (thread_index + 1 == count_threads)
    {
        step += (graph->order - step * count_threads);
    }

    for (int k = start; k < start + step; k++)
    {
        for (int i = 0; i < graph->order; i++)
        {
            for (int j = 0; j < graph->order; j++)
            {
                find_min_way(graph->matrix, i, j, k);
            }
        }
    }
}

void multithreading(int count_threads, graph_t* graph)
{
    std::vector<std::thread> threads(count_threads);

    for (int i = 0; i < count_threads; i++)
    {
        threads[i] = std::thread(parallel_floyd, std::ref(graph), count_threads, i);
    }

    for (int i = 0; i < count_threads; i++)
    {
        threads[i].join();
    }
}
\end{lstlisting}
\end{center}

\section{Функциональные тесты}

В таблице \ref{tbl:func_test} приведены функциональные тесты для функций, реализующих алгоритм Флойда. Все тесты пройдены успешно.

\begin{table}[h]
	\begin{center}
		\begin{threeparttable}
		\captionsetup{justification=raggedright,singlelinecheck=off}
		\caption{\label{tbl:func_test} Функциональные тесты}
		\begin{tabular}{|c|c|c|}
			\hline
			Матрица A & Кол-во потоков & Ожидаемый результат \\ 
			\hline
			$\begin{pmatrix}
				&
			\end{pmatrix}$ &
			4 &
			Сообщение об ошибке \\
			\hline

			$\begin{pmatrix}
				0 & 1 & -1 \\
				-2 & 0 & 3 \\
				4 & 5 & 0
			\end{pmatrix}$ &
			4 &
			Сообщение об ошибке \\
			\hline
 
			$\begin{pmatrix}
				0 & 1 & -1 \\
				-2 & 0 & 3 \\
				4 & 5 & 0
			\end{pmatrix}$ &
			0 &
			Сообщение об ошибке \\
			\hline

			$\begin{pmatrix}
				0 & 5 & 9 & -1 \\
				-1 & 0 & 2 & 8 \\
				-1 & -1 & 0 & 7 \\
				4 & -1 & -1 & 0 \\
			\end{pmatrix}$ &
			4 &
			$\begin{pmatrix}
				0 & 5 & 7 & 13 \\
				12 & 0 & 2 & 8 \\
				11 & 16 & 0 & 7 \\
				4 & 9 & 11 & 0 \\
			\end{pmatrix}$ \\
			\hline

		\end{tabular}
		\end{threeparttable}
	\end{center}
	
\end{table}

\section{Вывод}

Были реализованы функции последовательного и параллельного алгоритма Флойда. Было проведено функциональное тестирование указанных функций.